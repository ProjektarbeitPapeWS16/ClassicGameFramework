\section{Vorwort} %Worte zur Arbeit im Allgemeinen.

Dies ist eine Projektarbeit im Rahmen des Informatikstudiums an der Hochschule Karlsruhe - Technik und Wirtschaft. Sie wird betreut von Prof. Dr. Christian Pape. Das Thema ist "`Entwicklung eines Rahmenwerks zur Realisierung von Videospielklassikern"'. Die Arbeit ist aufgeteilt in die Entwicklung des Rahmenwerks und die Entwicklung der Videospiele Space Panic, Donkey Kong und Pac-Man. Wobei in diesem Dokument nur die Erstellung des Rahmenwerkes behandelt wird. Das Rahmenwerk wird in einer Zusammenarbeit der Studierenden Joan-Angelo Douvere, Christian Modery und Joachim Leiser in der Programmiersprache C++ und dem Framework OpenGL entwickelt. Die Videospielklassiker werden jeweils in Einzelarbeit mithilfe des entwickelten Rahmenwerks programmiert. Joan-Angelo Douvere implementiert Space Panic, Christian Modery Donkey Kong und Joachim Leiser Pac-Man. 