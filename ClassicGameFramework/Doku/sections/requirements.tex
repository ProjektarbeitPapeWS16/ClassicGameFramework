\section{Requirements}
Hier werden die Anforderungen des Rahmenwerks erläutert. Nachfolgendes soll das Rahmenwerk letztendlich leisten. Außerdem wird die gewünschte Schnittstelle für den Anwender näher betrachtet. 


\subsection{Ziel}
Das Ziel der Entwicklung des Rahmenwerkes ist es ein Werkzeug zur Verfügung zu stellen, mit dem die Entwicklung von Videospielklassikern wie Donkey Kong oder Space Panic erleichtert wird. Dabei ist die Entwicklung auf rasterbasierte zweidimensionale Spiele ausgelegt. 

Zum Erstellen von grafischen Anwendungen mit C++ kommt man um das Framework OpenGL kaum herum. Dieses Rahmenwerk soll dem Anwender jegliche Arbeit mit dieser Library abnehmen und speziell für Videospiele sinnvolle Schnittstellen anbieten, die die Komplexität von OpenGL kapseln. 

Das Rahmenwerk sollte also auf Spiele und somit sehr grafisch aufwendige Anwendungen ausgelegt sein. Es soll unnötige Komplexität kaschieren, aber dennoch genug Spielraum lassen um das Verhalten zu erweitern und zu manipulieren. Letztendlich soll ein Rahmenwerk immer die Arbeit des Entwicklers erleichtern.

Mithilfe des Rahmenwerks sollte also ein level- und rasterbasiertes Spiel mit einfacher Steuerung realisierbar sein. Damit sollten Texturen einfach angezeigt und manipuliert werden. Objekte sollten beweglich sein und ein Werkzeug um Kollisionen aufzuspüren sollte zur Verfügung gestellt werden. 


\subsection{Schnittstelle}
Da Anwendungen mit grafischen Oberflächen für gewöhnlich nach dem MVC Paradigma aufgebaut sind, sollte das Framework dieses auch unterstützen. Also sollte eine Klassenstruktur aufgebaut werden, deren Funktionalität durch Vererbung erweitert und angepasst werden kann. Dadurch wird der Anwender ermutigt sich an das Paradigma zu halten. 

Als Einstiegspunkt sollte ein Controller dienen. Dieser lenkt die Anwendung in die richtigen Bahnen. Er startet eine Initialisierungsroutine und stößt daraufhin das Spiel an.

Mit der View sollte es simpel sein ein Fenster zu erzeugen und einige Texturen mit Bitmaps anzuzeigen. Diese sollten regelmäßig aktualisiert werden. Für Tastatureingaben sollte eine Schnittstelle bereitgestellt werden, bei denen der Anwender eine Funktion bestimmten Tasten zuweisen kann, die bei ihrer Betätigung ausgeführt wird

Das Model soll eine Struktur für ein Spiel bereitstellen. Mit Session, Level und Entities (Objekten) soll eine levelartige Struktur bereitgestellt werden die für den Anwender intuitiv zu verwenden sind. Eine Session symbolisiert eine Sitzung, dass mehrere Level besitzt die nacheinander gespielt werden können. Die Level besitzen verschiedene Entities die miteinander interagieren und somit ein Spiellevel darstellen sollen. Die Entities symbolisieren Spielobjekte die miteinander interagieren und auf dem Bildschirm über Texturen dargestellt werden. Das Model kapselt diese Klassen und bringt sie in eine sinnvollen Ablauf. Desweiteren soll das Model eine Physik Engine enthalten die nach Kollisionen und relativen Posititionen von Entities abgefragt werden kann.